\documentclass[11pt]{beamer}
\usetheme{Madrid}
\usepackage[utf8]{inputenc}
\usepackage[german]{babel}
\usepackage[T1]{fontenc}
\usepackage{amsmath}
\usepackage{amsfonts}
\usepackage{amssymb}
\usepackage{graphicx}
\usepackage{multicol}
\usepackage{listings}
\usepackage{style/csharp}

\author{Florian Gehring}
\title{Vergleich zu C\#}
%\setbeamercovered{transparent} 
%\setbeamertemplate{navigation symbols}{} 
%\logo{} 
%\institute{} 
\date{18.06.2020} 
%\subject{} 
\begin{document}

% Create a line splitting two columns
\setlength{\columnseprule}{0.4pt}

\begin{frame}
\titlepage
\end{frame}

%\begin{frame}
%\tableofcontents
%\end{frame}

\begin{frame}{Geschichte}
\begin{itemize}
	\item Von Microsoft Entwickelt
	\item \glqq Direkter Konkurent\grqq{} zu Java
\end{itemize}
\end{frame}

\begin{frame}{Typen}
\begin{multicols}{2}
Java \\
\begin{itemize}
\item Primitive Typen nicht von \texttt{Object} abgeleitet
\item Call-by-Value, Call-By-Reference
\item Wrapper-Klasse \texttt{Integer} für \texttt{int}
\end{itemize}

\columnbreak


C\#\\
\begin{itemize}
\item Alles (auch \texttt{int}) von \texttt{object} abgeleitet
\item Zahlen, boolean sind \glqq{}Werttype\grqq{}
\item \texttt{int} kann mit \texttt{int?} Nullable gemacht werden
\end{itemize}
\end{multicols}
\end{frame}

\begin{frame}{Codebeispiele 1}


\end{frame}
\end{document}